\documentclass[12pt]{article}
\usepackage{tipa}
\begin{document}
\title{Linguistics 20, Homework 1}
\date{April 8th, 2019}
\author{Michael Wu\\UID: 404751542\\TA: Eleanor Glewwe\\Discussion 1F Friday 9:00-9:50 AM}
\maketitle

\section*{Chapter 1, Problem 3}

The words frall, sproke, and flube sound like acceptable English words. Mbood does not sound valid
because ``mb'' does not typically appear at the beginning of English words. Coofp does not sound valid
because ``fp'' does not typically appear at the end of English words. Ktleem does not sound valid
because ``ktl'' is not a common letter combination in English. Wordms does not sound valid
because ``dms'' does not typically appear at the end of English words. Bsarn does not sound valid
because ``bs'' does not typically appear at the beginning of English words.

\section*{Chapter 1, Problem 5}

\paragraph{a)}

Jason's mother left him with nothing to eat. I have changed ``himself'' to ``him'' since Jason's mother
is acting upon him, he is not acting upon himself.

\paragraph{b)}

Miriam is eager to talk to someone. The phrase ``talk to'' indicates that there should be a noun following the
final ``to'', so I have added the noun ``someone''.

\paragraph{c)}

This is an acceptable sentence.

\paragraph{d)}

This is an acceptable sentence.

\paragraph{e)}

Is the dog sleeping with the bone again? The verb ``sleeping'' does not take a noun as an argument so the only
way this sentence makes sense is to add the preposition ``with''.

\paragraph{f)}

This is an acceptable sentence.

\paragraph{g)}

This is an acceptable sentence.

\paragraph{h)}

This is an acceptable sentence.

\paragraph{i)}

Max cleaned it up. The verb phrase ``cleaned up'' should contain pronouns in the middle, so I have moved ``it'' in between the
verb phrase.

\paragraph{j)}

I hope you leave. I have changed the infinitive ``to leave'' to the verb form ``leave'' which matches the subject ``you''.

\paragraph{k)}

That you like liver surprises me. I have changed the verb ``likes'' to ``like'' in order to match the subject ``you''.

\section*{Chapter 1, Problem 6}

\paragraph{a)}

This sentence violates subject-verb agreement, as the correct form of the verb is ``doesn't''.

\paragraph{b)}

This sentence violates subject-verb agreement, as the correct form of the verb is ``were''.

\paragraph{c)}

This sentence violates subject-verb agreement, as the correct form of the verb is ``are''.

\paragraph{d)}

This sentence incorrectly uses the verb ``broke'' when the adjective ``broken'' should be used.

\paragraph{e)}

This sentence incorrectly uses ``me'' as the subject when it should use ``I''.

\paragraph{f)}

This sentence should use ``whom'' as the object of the preposition and keep the preposition together
so that it reads, ``With whom did you come?''.

\paragraph{g)}

This sentence violates subject-verb agreement, as the correct form of the verb is ``saw''.

\paragraph{h)}

This sentence incorrectly uses ``been'' for the present perfect tense when it should use ``has been''.

\paragraph{i)}

This sentence uses the verb ``cleaned'' as a noun when it should use the gerund ``cleaning''.

\paragraph{j)}

This sentence uses a double negative when it should use a single negative by saying ``doesn't have any''.

\paragraph{k)}

This sentence should use a singular pronoun ``his'' or ``her'' in order to agree with the subject ``somebody''.

\paragraph{l)}

This sentence should use the pronoun ``himself'' instead of ``hisself''.

Linguists would claim that these types of sentences are simply different dialects of English even though they violate
typical rules of normal English. They are not inherently wrong, they are simply atypical.

\section*{Chapter 2, Problem 1}

\paragraph{a)}

Two ways of pronouncing the letter \textless{}u\textgreater{} include /u/ like in \textless{}flute\textgreater{} and
/\textupsilon/ like in \textless{}put\textgreater{}.

\paragraph{b)}

Two examples where it is pronounced are \textless{}tough\textgreater{} and \textless{}trough\textgreater{}. Two examples
where it is not pronounced are \textless{}fought\textgreater{} and \textless{}bought\textgreater{}.

\paragraph{c)}

Two examples include \textless{}w\underline{ay}\textgreater{} and \textless{}p\underline{ai}n\textgreater{}.

\section*{Chapter 2, Problem 2}

\paragraph{a)}

There are two segments.

\paragraph{b)}

There are four segments.

\paragraph{c)}

There are four segments.

\paragraph{d)}

There are four segments.

\paragraph{e)}

There are ten segments.

\paragraph{f)}

There are thirteen segments.

\paragraph{g)}

There are four segments.

\paragraph{h)}

There are seven segments.

\section*{Chapter 2, Problem 3}

\paragraph{a)}

Voiced.

\paragraph{b)}

Voiced.

\paragraph{c)}

Voiced.

\paragraph{d)}

Voiceless.

\paragraph{e)}

Voiced.

\paragraph{f)}

Voiceless.

\paragraph{g)}

Voiceless.

\paragraph{h)}

Voiceless.

\paragraph{i)}

Voiceless.

\paragraph{j)}

Voiced.

\paragraph{k)}

Voiced.

\paragraph{l)}

Voiced.

\paragraph{m)}

Voiced.

\paragraph{n)}

Voiced.

\paragraph{o)}

Voiced.

\paragraph{p)}

Voiced.

\section*{Problem C}

Three misconceptions about language are that a person's inability to speak a standardized form of a language indicates stupidity,
biological factors cause people to more easily learn specific languages, and changes to a language are inherently bad and lead to their deterioration.

The first misconception stems from thinking that there is an objectively correct way to speak a given language, and variations
from the norm indicate stupidity. For example, blacks who speak using Ebonics may be perceived as stupid. In reality Ebonics is a dialect of
English with its own consistent grammar. Its speakers use it because they had to in order to be understood within their community. Although
speaking Ebonics may correlate with having below average intelligence, this may stem from factors such as poverty and lack of education within these
communities rather than from properties of the Ebonics language. It is entirely possible that an individual with above average intelligence speaks Ebonics.

An example of the second misconception is thinking that Mexicans learn Spanish easier while Germans learn German easier due to differences in DNA.
In reality, most people's language acquisition skills are fairly similar and an ethnic Mexican would learn German just as well as a native German
if they both grew up in the same community. Language acquisition is mainly a function of the environment that people live in, as they must learn
the dominant language in their communities in order to communicate effectively. There is no evidence to support the idea that genetics causes certain
ethnicities to be more suited to one language or another.

The last misconception is that changes to a language over time are bad, such as when people complain that the introduction of new slang and
grammar constructs into a language cause deterioration to a language. While it is understandable for these people to feel frustrated
at not understanding new terms or sentences, in reality languages change constantly. As long as new cultures and ideas emerge, people
will come up with new language to describe these things. This happens almost everywhere in areas such as in politics, music, television, science,
sports, and technology.

\section*{Problem D}

To linguists, grammar means the structure that allows people to understand a language. Three universal properties of grammar are universality,
parity, and mutability. Universality means that grammars for human language tend to share certain properties.
For example every language splits noises into discrete words and they tend to order nouns and verbs in the same way.
Parity means that all grammars have equal importance, and no grammar is superior to another. Any grammar that is widely used must be comprehensible,
and so if two different grammars both produce comprehensible statements they are equally useful. Mutability means that grammar changes over time.
As people learn a language, they can coin new phrases and syntax. Over long periods of time this eventually causes languages to become very different.
Since there is no single authority on what is correct in everyday language use, nothing can stop this change.

\end{document}